\documentclass[10pt, a4paper, onecolumn, oneside, titlepage, openany]{book}

\usepackage[T1]{fontenc}
\usepackage[utf8]{inputenc}
\usepackage{titlesec} % chapter on line with text
\usepackage{fancyhdr}
\usepackage{hyperref}
\hypersetup{
    colorlinks=true,
    linkcolor=blue, %\ref{}
    filecolor=magenta, % \href{run:./file.txt}{File.txt}
    urlcolor=cyan, % \href{http://www.overleaf.com}{Link}
    pdftitle={Cisco CCNA}
    pdfpagemode=FullScreen,
    }
\usepackage{menukeys}
\usepackage{fancyvrb}

% table formatting
\setlength{\arrayrulewidth}{0.5mm} % border
\setlength{\tabcolsep}{18pt} % space from border (X axis)
\renewcommand{\arraystretch}{1.5} % space from border (Y axis)
\usepackage[format=hang,font=small,labelfont=bf]{caption} % bold table caption

% chapter formatting
\renewcommand{\chaptername}{}
\titleformat{\chapter}[hang]{\normalfont\huge\bfseries}{\chaptertitlename\ \thechapter.}{1em}{}

% decorative lines:
\renewcommand{\headrulewidth}{2pt}
\renewcommand{\footrulewidth}{2pt}
\pagestyle{fancy}
\fancyhf{}
    \chead{\leftmark}
    \cfoot{\thepage}
\fancypagestyle{plain}{
\fancyhf{}
    \chead{\leftmark}
    \cfoot{\thepage}
}

% verbatim formatting
\definecolor{device}{RGB}{0, 100, 200}
\definecolor{command}{RGB}{41, 182, 0}
\definecolor{nocommand}{RGB}{222, 0, 0}
\definecolor{showcommand}{RGB}{255, 80, 0}
\definecolor{warncommand}{RGB}{255, 80, 0}
\definecolor{root}{RGB}{222, 0, 0}
\definecolor{user}{RGB}{0, 150, 0}
\definecolor{dir}{RGB}{0, 100, 200}
\definecolor{file}{RGB}{77, 187, 101}
\definecolor{comment}{RGB}{0, 182, 182}
\definecolor{background}{RGB}{240, 240, 240}
\renewcommand{\FancyVerbFormatLine}[1]{\colorbox{background}{#1}}





\title{\textbf{Cisco CCNA}}
\author{AISK11}
\date{March 2022}

\begin{document}
\maketitle
\tableofcontents

\chapter{Basics}
\section{Shortcuts}
\begin{table}[h!]
\centering
\begin{tabular}{|c|c|}
    \hline
    \textbf{Keys} & \textbf{Description} \\
    \hline
    \keys{Ctrl + Shift + 6} & Interrupt current IOS process \\
    \keys{Ctrl + R} & Reprint current command line \\
    \hline
\end{tabular}
\caption{Cisco IOS shortcuts.}
\label{table:1}
\end{table}

\section{Help}
\begin{itemize}
    \item \textbf{Show available commands:}
\begin{Verbatim}[commandchars=\\\{\}]
\textcolor{device}{Device*} \textcolor{showcommand}{?}
\end{Verbatim}
    \item \textbf{Show available command options:}
\begin{Verbatim}[commandchars=\\\{\}]
\textcolor{device}{Device*} <COMMAND> \textcolor{showcommand}{?}
\end{Verbatim}
\end{itemize}

\section{CLI Navigation}
\begin{itemize}
    \item \textbf{Privilege escalation}
    \begin{itemize}
        \item \textbf{Access privileged mode:}
\begin{Verbatim}[commandchars=\\\{\}]
\textcolor{device}{Device>} \textcolor{command}{enable}
\end{Verbatim}
        \item \textbf{Access configuration mode:}
\begin{Verbatim}[commandchars=\\\{\}]
\textcolor{device}{Device#} \textcolor{command}{configure terminal}
\end{Verbatim}
    \end{itemize}
    \item \textbf{Privilege deescalation}
    \begin{itemize}
        \item \textbf{Return to less privileged mode:}
\begin{Verbatim}[commandchars=\\\{\}]
\textcolor{device}{Device*} \textcolor{nocommand}{exit}
\end{Verbatim}
    \end{itemize}
\end{itemize}

\section{Hostname}
\label{hostname}
\begin{itemize}
    \item \textbf{Set hostname on device:}
\begin{Verbatim}[commandchars=\\\{\}]
\textcolor{device}{Device(config)#} \textcolor{command}{hostname} <HOSTNAME>
\end{Verbatim}
\end{itemize}

\section{Configuration}
\subsection{Memory Hardware}
\begin{table}[h!]
\centering
\begin{tabular}{|c|c|}
    \hline
    \textbf{Memory} & \textbf{Description} \\
    \hline
    ROM & POST + bootloader \\
    Flash & Cisco IOS \\
    NVRAM & startup-config \\
    RAM & running-config \\
    \hline
\end{tabular}
\caption{Cisco memory types overview.}
\label{table:2}
\end{table}
\subsection{Boot Order}
\begin{enumerate}
    \item Device performs POST in ROM.
    \item On successful POST, bootloader is copied from ROM to RAM.
    \item Bootloader chooses IOS image from Flash to RAM and loads IOS image.
    \item IOS runs startup-config from NVRAM.
\end{enumerate}
\subsection{Configuration}
\begin{itemize}
    \item \textbf{Save configuration (RAM -> NVRAM):}
\begin{Verbatim}[commandchars=\\\{\}]
\textcolor{device}{Device#} \textcolor{command}{copy running-config startup-config}
\end{Verbatim}
    \item \textbf{Load configuration (NVRAM -> RAM):}
\begin{Verbatim}[commandchars=\\\{\}]
\textcolor{device}{Device#} \textcolor{nocommand}{copy startup-config running-config}
\end{Verbatim}
    \item \textbf{Erase full startup configuration}
    \begin{enumerate}
        \item \textbf{See all files:}
\begin{Verbatim}[commandchars=\\\{\}]
\textcolor{device}{Device#} \textcolor{showcommand}{dir}
\end{Verbatim}
        \item \textbf{Remove startup-config (NVRAM):}
\begin{Verbatim}[commandchars=\\\{\}]
\textcolor{device}{Device#} \textcolor{nocommand}{erase startup-config}
\end{Verbatim}
        \item \textbf{Remove VLANs data:}
\begin{Verbatim}[commandchars=\\\{\}]
\textcolor{device}{Switch#} \textcolor{nocommand}{delete vlan.dat}
\end{Verbatim}
        \item \textbf{Reload device:}
\begin{Verbatim}[commandchars=\\\{\}]
\textcolor{device}{Device#} \textcolor{nocommand}{reload}
\end{Verbatim}
    \end{enumerate}
\end{itemize}

\chapter{Passwords}
\section{Banner}
\subsection{MOTD}
Used to display temporary messages.
\begin{itemize}
    \item \textbf{Set MOTD:}
\begin{Verbatim}[commandchars=\\\{\}]
\textcolor{device}{Device(config)#} \textcolor{command}{banner motd} \%<TEXT>\%
\end{Verbatim}
    \item \textbf{Remove MOTD:}
\begin{Verbatim}[commandchars=\\\{\}]
\textcolor{device}{Device(config)#} \textcolor{nocommand}{no banner motd} \%<TEXT>\%
\end{Verbatim}
\end{itemize}

\section{Local}
\subsection{User EXEC mode}
\begin{itemize}
    \item \textbf{Console access (Plaintext):}
\begin{Verbatim}[commandchars=\\\{\}]
\textcolor{device}{Device(config)#} \textcolor{command}{line} console 0
\textcolor{device}{Device(config-line)#} \textcolor{command}{password} <PASSWORD>
\textcolor{device}{Device(config-line)#} \textcolor{command}{login}
\end{Verbatim}
    \item \textbf{VTY (SSH and Telnet) access (Plaintext):}
    \newline Telnet is auto enabled after VTY password is configured.
\begin{Verbatim}[commandchars=\\\{\}]
\textcolor{device}{Device(config)#} \textcolor{command}{line} vty 0 15
\textcolor{device}{Device(config-line)#} \textcolor{command}{password} <PASSWORD>
\textcolor{device}{Device(config-line)#} \textcolor{command}{login}
\end{Verbatim}
\end{itemize}
\subsection{Privileged EXEC mode}
\begin{itemize}
    \item \textbf{Privilege EXEC access (MD5):}
\begin{Verbatim}[commandchars=\\\{\}]
\textcolor{device}{Device(config)#} \textcolor{command}{enable secret} <PASSWORD>
\end{Verbatim}
\end{itemize}
\subsection{Encrypt Passwords}
\begin{itemize}
    \item \textbf{Encrypt all Plaintext passwords with Vigenere Cipher:}
\begin{Verbatim}[commandchars=\\\{\}]
\textcolor{device}{Device(config)#} \textcolor{command}{service password-encryption}
\end{Verbatim}
\end{itemize}
\subsection{Advanced Password Settings}
\begin{itemize}
    \item \textbf{Minimal password length (8):}
\begin{Verbatim}[commandchars=\\\{\}]
\textcolor{device}{Device(config)#} \textcolor{command}{security} passwords min-length <8>
\end{Verbatim}
    \item \textbf{Block login for 120s after 3 failed attempts within 60 seconds:}
\begin{Verbatim}[commandchars=\\\{\}]
\textcolor{device}{Device(config)#} \textcolor{command}{login} block-for <120> attempts <3> within <60>
\end{Verbatim}
    \item \textbf{Logout from exec mode after 5 minutes and 30 seconds of inactivity:}
\begin{Verbatim}[commandchars=\\\{\}]
\textcolor{device}{Device(config-line)#} \textcolor{command}{exec-timeout} <5> <30>
\end{Verbatim}
\end{itemize}

\section{SSH}
\begin{enumerate}
    \item \textbf{Requires IP address:}
    \begin{itemize}
        \item \textbf{Switch:}
        \newline See section \ref{management_vlan}
        \item \textbf{Router:}
        \newline See section \ref{routing}
    \end{itemize}
    \item \textbf{Requires Hostname:}
    \newline See section \ref{hostname}
    \item \textbf{Set default domain name:}
\begin{Verbatim}[commandchars=\\\{\}]
\textcolor{device}{Device(config)#} \textcolor{command}{ip domain-name} <DOMAIN.COM>
\end{Verbatim}
    \item \textbf{Generate encryption keys:}
    Recommended at least 1024 bits.
\begin{Verbatim}[commandchars=\\\{\}]
\textcolor{device}{Device(config)#} \textcolor{command}{crypto} key generate rsa general-keys modulus <360-2048>
\end{Verbatim}
    \item \textbf{Create login for user:}
\begin{Verbatim}[commandchars=\\\{\}]
\textcolor{device}{Device(config)#} \textcolor{command}{username} <USER> secret <PASSWORD>
\end{Verbatim}
    \item \textbf{Use login with specified username and passwords:}
    \newline Overwrites previously set up with only VTY passwords (without username).
\begin{Verbatim}[commandchars=\\\{\}]
\textcolor{device}{Device(config)#} \textcolor{command}{line} vty 0 15
\textcolor{device}{Device(config-line)#} \textcolor{command}{login local}
\end{Verbatim}
    \item \textbf{Disable telnet:}
\begin{Verbatim}[commandchars=\\\{\}]
\textcolor{device}{Device(config-line)#} \textcolor{command}{transport input ssh}
\end{Verbatim}
    \item \textbf{Enable more secure SSHv2:}
\begin{Verbatim}[commandchars=\\\{\}]
\textcolor{device}{Device(config)#} \textcolor{command}{ip ssh version} 2
\end{Verbatim}
\end{enumerate}
\begin{itemize}
    \item \textbf{Troubleshooting}
    \begin{itemize}
        \item \textbf{SSH config (version, authentication):}
\begin{Verbatim}[commandchars=\\\{\}]
\textcolor{device}{Device#} \textcolor{showcommand}{show ip ssh}
\end{Verbatim}
        \item \textbf{SSH sessions:}
\begin{Verbatim}[commandchars=\\\{\}]
\textcolor{device}{Device#} \textcolor{showcommand}{show ssh}
\end{Verbatim}
    \end{itemize}
\end{itemize}

\chapter{VLANs}
\section{VLANs}
\textbf{Default VLAN:} VLAN 1, cannot be renamed nor deleted.
\newline \textbf{Native VLAN:} Used by untagged traffic. Default VLAN1.
\newline \textbf{Management VLAN:} Used for management traffic (Telnet, SSH, HTTP(S), SNMP). Default VLAN1.
\newline \textbf{Data VLAN:} User-generated traffic.
\newline \textbf{Voice VLAN:} Transmission priority over other types of network traffic. Ability to be routed around congested areas on the network. Delay less than 150ms across the network.
\begin{itemize}
    \item \textbf{Normal Range:}
    \begin{itemize}
        \item 1 - 1005
        \item Reserved: 1, 1002-1005
    \end{itemize}
    \item \textbf{Extended Range:}
    \begin{itemize}
        \item 1006 - 4094
        \item They support fewer VLAN features than normal VLANs.
        \item Required VTP transparent mode config mode to support extended range.
    \end{itemize}
\end{itemize}
\subsection{Extended VLAN Range}
\begin{enumerate}
    \item \textbf{Enable extended VLAN Range:}
\begin{Verbatim}[commandchars=\\\{\}]
\textcolor{device}{Switch(config)#} \textcolor{command}{vtp} mode transparent
\end{Verbatim}
\end{enumerate}
\subsection{Management VLAN}
\label{management_vlan}
\begin{enumerate}
    \item \textbf{Create Management VLAN:}
\begin{Verbatim}[commandchars=\\\{\}]
\textcolor{device}{Switch(config)#} \textcolor{command}{vlan} <2>
\textcolor{device}{Switch(config-vlan)#} \textcolor{command}{name} <Management>
\end{Verbatim}
    \item \textbf{Assign IP address to VLAN:}
\begin{Verbatim}[commandchars=\\\{\}]
\textcolor{device}{Switch(config)#} \textcolor{command}{interface} vlan <2>
\textcolor{device}{Switch(config-if)#} \textcolor{command}{ip} address <192.168.2.2> <255.255.255.0>
\textcolor{device}{Switch(config-if)#} \textcolor{command}{no} shutdown
\end{Verbatim}
    \item \textbf{Assign default gateway:}
\begin{Verbatim}[commandchars=\\\{\}]
\textcolor{device}{Switch(config)#} \textcolor{command}{ip} default-gateway <192.168.2.1>
\end{Verbatim}
\end{enumerate}
\subsection{Access VLAN}
Configured on interfaces connected to end devices.
\begin{enumerate}
    \item \textbf{Create VLAN:}
    \begin{itemize}
        \item \textbf{Data VLAN:}
\begin{Verbatim}[commandchars=\\\{\}]
\textcolor{device}{Switch(config)#} \textcolor{command}{vlan} <10>
\textcolor{device}{Switch(config-vlan)#} \textcolor{command}{name} <Data>
\end{Verbatim}
        \item \textbf{Voice VLAN:}
\begin{Verbatim}[commandchars=\\\{\}]
\textcolor{device}{Switch(config)#} \textcolor{command}{vlan} <150>
\textcolor{device}{Switch(config-vlan)#} \textcolor{command}{name} <Voice>
\end{Verbatim}
    \end{itemize}
    \item \textbf{Assign VLAN to interface:}
\begin{Verbatim}[commandchars=\\\{\}]
\textcolor{device}{Switch(config)#} \textcolor{command}{interface} [range] <FastEthernet0/1> [- 24]
\textcolor{device}{Switch(config-if)#} \textcolor{command}{switchport} mode <access>
\end{Verbatim}
    \begin{itemize}
        \item \textbf{Data VLAN:}
\begin{Verbatim}[commandchars=\\\{\}]
\textcolor{device}{Switch(config-if)#} \textcolor{command}{switchport} access vlan <10>
\end{Verbatim}
        \item \textbf{Voice VLAN:}
\begin{Verbatim}[commandchars=\\\{\}]
\textcolor{device}{Switch(config-if)#} \textcolor{command}{mls} qos trust cos
\textcolor{device}{Switch(config-if)#} \textcolor{command}{switchport} voice vlan <150>
\end{Verbatim}
    \end{itemize}
\end{enumerate}
\subsection{Trunk VLAN}
Configured on interfaces connected to other switch or router.
\begin{enumerate}
    \item \textbf{Configure interface as trunk:}
\begin{Verbatim}[commandchars=\\\{\}]
\textcolor{device}{Switch(config)#} \textcolor{command}{interface} <GigabitEthernet 0/1>
\textcolor{device}{Switch(config)#} \textcolor{command}{switchport} trunk encapsulation dot1q
\textcolor{device}{Switch(config-if)#} \textcolor{command}{switchport} mode <trunk>
\textcolor{device}{Switch(config-if)#} \textcolor{command}{switchport} nonegotiate
\end{Verbatim}
    \item \textbf{Change native VLAN:}
\begin{Verbatim}[commandchars=\\\{\}]
\textcolor{device}{Switch(config-if)#} \textcolor{command}{switchport} trunk native vlan <1000>
\textcolor{device}{Switch(config-if)#} \textcolor{nocommand}{no switchport trunk native vlan}
\end{Verbatim}
    \item \textbf{Select which VLANs will be propagated:}
\newline If none are selected, then all VLANs are propagated.
\begin{Verbatim}[commandchars=\\\{\}]
\textcolor{device}{Switch(config-if)#} \textcolor{command}{switchport} trunk allowed vlan <2,10,150,1000>
\textcolor{device}{Switch(config-if)#} \textcolor{nocommand}{no switchport trunk allowed vlan}
\end{Verbatim}
\end{enumerate}

\section{Inter-VLAN Routing}
Requires default GW IP addresses on router. See section \ref{routing}.
\subsection{Router on a Stick}
\begin{enumerate}
    \item \textbf{Management VLAN config:}
\begin{Verbatim}[commandchars=\\\{\}]
\textcolor{device}{Router(config)#} \textcolor{command}{interface} <G0/0/1.10>
\textcolor{device}{Router(config-subif)#} \textcolor{command}{description} <Default Gateway for VLAN 2>
\textcolor{device}{Router(config-subif)#} \textcolor{command}{encapsulation} dot1Q <2>
\textcolor{device}{Router(config-subif)#} \textcolor{command}{ip} add <192.168.2.1> <255.255.255.0>
\end{Verbatim}
    \item \textbf{Data VLAN config:}
\begin{Verbatim}[commandchars=\\\{\}]
\textcolor{device}{Router(config)#} \textcolor{command}{interface} <G0/0/1.10>
\textcolor{device}{Router(config-subif)#} \textcolor{command}{description} <Default Gateway for VLAN 10>
\textcolor{device}{Router(config-subif)#} \textcolor{command}{encapsulation} dot1Q <10>
\textcolor{device}{Router(config-subif)#} \textcolor{command}{ip} add <192.168.10.1> <255.255.255.0>
\end{Verbatim}
    \item \textbf{Voice VLAN config:}
\begin{Verbatim}[commandchars=\\\{\}]
\textcolor{device}{Router(config)#} \textcolor{command}{interface} <G0/0/1.150>
\textcolor{device}{Router(config-subif)#} \textcolor{command}{description} <Default Gateway for VLAN 150>
\textcolor{device}{Router(config-subif)#} \textcolor{command}{encapsulation} dot1Q <150>
\textcolor{device}{Router(config-subif)#} \textcolor{command}{ip} add <192.168.150.1> <255.255.255.0>
\end{Verbatim}
    \item \textbf{Configure interface itself:}
\begin{Verbatim}[commandchars=\\\{\}]
\textcolor{device}{Router(config)#} \textcolor{command}{interface} <G0/0/1>
\textcolor{device}{Router(config-if)#} \textcolor{command}{description} <Trunk link to Switch>
\textcolor{device}{Router(config-if)#} \textcolor{command}{no} shut
\end{Verbatim}
\end{enumerate}
\subsection{L3 Switch}
\begin{enumerate}
    \item \textbf{Create VLANs}
    \begin{itemize}
        \item \textbf{Management}
\begin{Verbatim}[commandchars=\\\{\}]
\textcolor{device}{L3Switch(config)#} \textcolor{command}{vlan} <2>
\textcolor{device}{L3Switch(config-vlan)#} \textcolor{command}{name} <Management>
\end{Verbatim}
        \item \textbf{Data}
\begin{Verbatim}[commandchars=\\\{\}]
\textcolor{device}{L3Switch(config)#} \textcolor{command}{vlan} <10>
\textcolor{device}{L3Switch(config-vlan)#} \textcolor{command}{name} <Data>
\end{Verbatim}
        \item \textbf{Voice}
\begin{Verbatim}[commandchars=\\\{\}]
\textcolor{device}{L3Switch(config)#} \textcolor{command}{vlan} <150>
\textcolor{device}{L3Switch(config-vlan)#} \textcolor{command}{name} <Voice>
\end{Verbatim}
    \end{itemize}
        \item \textbf{Create Gateway for VLANs:}
\begin{itemize}
    \item \textbf{Management VLAN GW:}
\begin{Verbatim}[commandchars=\\\{\}]
\textcolor{device}{L3Switch(config)#} \textcolor{command}{interface} vlan <2>
\textcolor{device}{L3Switch(config-if)#} \textcolor{command}{description} <Default Gateway for VLAN 2>
\textcolor{device}{L3Switch(config-if)#} \textcolor{command}{ip} address <192.168.2.1> <255.255.255.0>
\textcolor{device}{L3Switch(config-if)#} \textcolor{command}{no} shut
\end{Verbatim}
    \item \textbf{Data VLAN GW:}
\begin{Verbatim}[commandchars=\\\{\}]
\textcolor{device}{L3Switch(config)#} \textcolor{command}{interface} vlan <10>
\textcolor{device}{L3Switch(config-if)#} \textcolor{command}{description} <Default Gateway for VLAN 10>
\textcolor{device}{L3Switch(config-if)#} \textcolor{command}{ip} address <192.168.10.1> <255.255.255.0>
\textcolor{device}{L3Switch(config-if)#} \textcolor{command}{no} shut
\end{Verbatim}
    \item \textbf{Voice VLAN GW:}
\begin{Verbatim}[commandchars=\\\{\}]
\textcolor{device}{L3Switch(config)#} \textcolor{command}{interface} vlan <150>
\textcolor{device}{L3Switch(config-if)#} \textcolor{command}{description} <Default Gateway for VLAN 150>
\textcolor{device}{L3Switch(config-if)#} \textcolor{command}{ip} address <192.168.150.1> <255.255.255.0>
\textcolor{device}{L3Switch(config-if)#} \textcolor{command}{no} shut
\end{Verbatim}
\end{itemize}
    \item \textbf{Configure Access Ports}
\begin{Verbatim}[commandchars=\\\{\}]
\textcolor{device}{L3Switch(config)#} \textcolor{command}{interface} <FastEthernet 0/1>
\textcolor{device}{L3Switch(config-if)#} \textcolor{command}{description} <Acces port to PC>
\textcolor{device}{L3Switch(config-if)#} \textcolor{command}{switchport} mode <access>
\textcolor{device}{L3Switch(config-if)#} \textcolor{command}{switchport} access vlan <10>
\textcolor{device}{L3Switch(config-if)#} \textcolor{command}{mls} qos trust cos
\textcolor{device}{L3Switch(config-if)#} \textcolor{command}{switchport} voice vlan <150>
\end{Verbatim}
    \item \textbf{Enable IP routing:}
\begin{Verbatim}[commandchars=\\\{\}]
\textcolor{device}{L3Switch(config)#} \textcolor{command}{ip} routing
\end{Verbatim}
\end{enumerate}




\chapter{Routing}
\label{routing}
\section{Static}
\begin{verbatim}
+--------------------+                                      +--------------------+
|LAN 1               |G0/0/0+----+G0/0/1  G0/0/0+----+G0/0/1|LAN 2               |
|10.0.1.0/24         |------| R1 |--------------| R2 |------|192.168.0.0/24      |
|2001:db8:acad:1::/64|    .1+----+.1          .2+----+.1    |2001:db8:acad:2::/64|
+--------------------+            172.16.0.0/30             +--------------------+
                              2001:db8:acad:0::/64
IPv6 LLC Addresses:
R1 G/0/0/0: fe80::1:1/64
R1 G/0/0/1: fe80::1:1/64
R2 G/0/0/0: fe80::1:2/64
R2 G/0/0/1: fe80::1:1/64
\end{verbatim}
\subsection{IPv4}
\begin{enumerate}
    \item \textbf{Assign IP address to interfaces:}
\begin{Verbatim}[commandchars=\\\{\}]
\textcolor{device}{R1(config)#} \textcolor{command}{interface} <G0/0/0>
\textcolor{device}{R1(config-if)#} \textcolor{command}{ip} address <10.0.1.1> <255.255.255.0>
\textcolor{device}{R1(config-if)#} \textcolor{command}{no shut}
\textcolor{device}{R1(config)#} \textcolor{command}{interface} <G0/0/1>
\textcolor{device}{R1(config-if)#} \textcolor{command}{ip} address <172.16.0.1> <255.255.255.252>
\textcolor{device}{R1(config-if)#} \textcolor{command}{no shut}
\end{Verbatim}
    \item \textbf{Static Route:}
    \begin{itemize}
        \item \textbf{Normal Static Route:}
\begin{Verbatim}[commandchars=\\\{\}]
\textcolor{device}{R1(config)#} \textcolor{command}{ip} route <192.168.0.0> <255.255.255.0>
<172.16.0.2|G0/0/1>
\end{Verbatim}
        \item \textbf{Fully Specified Static Route:}
\begin{Verbatim}[commandchars=\\\{\}]
\textcolor{device}{R1(config)#} \textcolor{command}{ip} route <192.168.0.0> <255.255.255.0>
<G0/0/1 172.16.0.2>
\end{Verbatim}
        \end{itemize}
    \item \textbf{Default Static Route:}
\begin{Verbatim}[commandchars=\\\{\}]
\textcolor{device}{R1(config)#} \textcolor{command}{ip} route 0.0.0.0 0.0.0.0 <172.16.0.2|G0/0/1>
\end{Verbatim}
    \item \textbf{Floating Static Route:}
\newline Appears in routing table after primary route fails.
\newline Static routes have distance of \textbf{1}. Less is preferable.
\begin{Verbatim}[commandchars=\\\{\}]
\textcolor{device}{R1(config)#} \textcolor{command}{ip} route 0.0.0.0 0.0.0.0 <IPv4|INTERFACE> [5]
\end{Verbatim}
\end{enumerate}
\subsection{IPv6}
\label{Router_IPv6}
\begin{enumerate}
    \item \textbf{Enable router as IPv6:}
\begin{Verbatim}[commandchars=\\\{\}]
\textcolor{device}{R1(config)#} \textcolor{command}{ipv6} unicast-routing
\end{Verbatim}
    \item \textbf{Assign Link-Local Address (LLC):}
\newline Has to be unique only within the LAN.
\begin{Verbatim}[commandchars=\\\{\}]
\textcolor{device}{R1(config)#} \textcolor{command}{interface} <G0/0/0>
\textcolor{device}{R1(config-if)#} \textcolor{command}{ipv6} address <fe80::1:1>/64 link-local
\textcolor{device}{R1(config-if)#} \textcolor{command}{no shutdown}
\textcolor{device}{R1(config)#} \textcolor{command}{interface} <G0/0/1>
\textcolor{device}{R1(config-if)#} \textcolor{command}{ipv6} address <fe80::1:1>/64 link-local
\textcolor{device}{R1(config-if)#} \textcolor{command}{no shutdown}
\end{Verbatim}
    \item \textbf{Assign Global Unicast Address (GUA):}
\begin{Verbatim}[commandchars=\\\{\}]
\textcolor{device}{R1(config)#} \textcolor{command}{interface} <G0/0/0>
\textcolor{device}{R1(config-if)#} \textcolor{command}{ipv6} address <2001:db8:acad:1::1>/64
\textcolor{device}{R1(config-if)#} \textcolor{command}{no shutdown}
\textcolor{device}{R1(config)#} \textcolor{command}{interface} <G0/0/1>
\textcolor{device}{R1(config-if)#} \textcolor{command}{ipv6} address <2001:db8:acad:0::1>/64
\textcolor{device}{R1(config-if)#} \textcolor{command}{no shutdown}
\end{Verbatim}
    \item \textbf{Static Route:}
    \begin{itemize}
        \item \textbf{Normal Static Route:}
\newline Can also be specified with LLC IP address but please avoid.
\begin{Verbatim}[commandchars=\\\{\}]
\textcolor{device}{R1(config)#} \textcolor{command}{ipv6} route <2001:db8:acad:2::/64>
<2001:db8:acad:0::2|G0/0/1>
\end{Verbatim}
        \item \textbf{Fully Specified Static Route}
\begin{Verbatim}[commandchars=\\\{\}]
\textcolor{device}{R1(config)#} \textcolor{command}{ipv6} route <2001:db8:acad:2::/64>
<G0/0/1 2001:db8:acad:0::2>
\end{Verbatim}
    \item \textbf{Default Static Route:}
\begin{Verbatim}[commandchars=\\\{\}]
\textcolor{device}{R1(config)#} \textcolor{command}{ipv6} route ::/0 <2001:db8:acad:2|G0/0/1>
\end{Verbatim}
    \end{itemize}
    \item \textbf{Floating Static Route:}
\newline Appears in routing table after primary route fails.
\newline Static routes have distance of \textbf{1}. Less preferable
\begin{Verbatim}[commandchars=\\\{\}]
\textcolor{device}{R1(config)#} \textcolor{command}{ipv6} route ::/0 <2001:db8:acad:2|G0/0/1> [5]
\end{Verbatim}
\end{enumerate}

\section{OSPF}
\textbf{Router ID:} explicitly configured OR highest loopback OR highest IP address.
\textbf{OSPF Election:} highest Router ID is elected.
\begin{enumerate}
    \item \textbf{Enable OSPFv2:}
\newline Process ID does not have to be the same on all routers, but is recommended.
\begin{Verbatim}[commandchars=\\\{\}]
\textcolor{device}{R1(config)#} \textcolor{command}{router} ospf <1>
\end{Verbatim}
    \item \textbf{Configure Router ID:}
    \begin{itemize}
        \item \textbf{Router ID:}
\begin{Verbatim}[commandchars=\\\{\}]
\textcolor{device}{R1(config-router)#} \textcolor{command}{router} router-id <1.1.1.1>
\end{Verbatim}
    \item \textbf{Loopback:}
\begin{Verbatim}[commandchars=\\\{\}]
\textcolor{device}{R1(config)#} \textcolor{command}{interface} Loopback <1>
\textcolor{device}{R1(config-if)#} \textcolor{command}{ip} address <1.1.1.1> <255.255.255.255>
\end{Verbatim}
    \item \textbf{Verify Router ID:}
\begin{Verbatim}[commandchars=\\\{\}]
\textcolor{device}{R1#} \textcolor{showcommand}{show ip protocols | include Router ID}
\end{Verbatim}
    \end{itemize}
    \item \textbf{Modify Router ID:}
\begin{Verbatim}[commandchars=\\\{\}]
\textcolor{device}{R1(config)#} \textcolor{command}{router} ospf <1>
\textcolor{device}{R1(config-router)#} \textcolor{command}{router} router-id <1.1.1.1>
\textcolor{device}{R1#} \textcolor{command}{clear ip ospf process}
\end{Verbatim}
\end{enumerate}



\chapter{Services}
\section{DHCP}
Requires IP address for router. See section \ref{routing}.
\subsection{DHCPv4}
\begin{enumerate}
    \item \textbf{Exclude IPv4 Addresses (range):}
\begin{Verbatim}[commandchars=\\\{\}]
\textcolor{device}{Router(config)#} \textcolor{command}{ip} dhcp excluded-address <192.168.1.1 [192.168.1.9]>
\textcolor{device}{Router(config)#} \textcolor{command}{ip} dhcp excluded-address <192.168.1.254>
\end{Verbatim}
    \item \textbf{Define pool name:}
\begin{Verbatim}[commandchars=\\\{\}]
\textcolor{device}{Router(config)#} \textcolor{command}{ip} dhcp pool <DHCP-POOL-1>
\end{Verbatim}
    \item \textbf{Define network range:}
\begin{Verbatim}[commandchars=\\\{\}]
\textcolor{device}{Router(dhcp-config)#} \textcolor{command}{network} <192.168.1.0 255.255.255.0>
\end{Verbatim}
    \item \textbf{Define default GW:}
\begin{Verbatim}[commandchars=\\\{\}]
\textcolor{device}{Router(dhcp-config)#} \textcolor{command}{default-router} <192.168.1.1>
\end{Verbatim}
    \item \textbf{Define DNS server:}
\begin{Verbatim}[commandchars=\\\{\}]
\textcolor{device}{Router(dhcp-config)#} \textcolor{command}{dns-server} <1.1.1.1>
\end{Verbatim}
    \item \textbf{Change duration of DHCP lease:}
\begin{Verbatim}[commandchars=\\\{\}]
\textcolor{device}{Router(dhcp-config)#} \textcolor{command}{lease} <DAYS [HOURS [MINUTES]]>
\end{Verbatim}
    \item \textbf{Configure domain-name DHCP lease:}
\begin{Verbatim}[commandchars=\\\{\}]
\textcolor{device}{Router(dhcp-config)#} \textcolor{command}{domain-name} <domain.com>
\end{Verbatim}
    \item \textbf{Verify}
\begin{Verbatim}[commandchars=\\\{\}]
\textcolor{device}{Router#} \textcolor{showcommand}{show} ip dhcp binding
\end{Verbatim}
    \item \textbf{Enable/Disable}
\newline Default: Enabled
\begin{Verbatim}[commandchars=\\\{\}]
\textcolor{device}{Router(config)#} \textcolor{command}{service} dhcp
\textcolor{device}{Router(config)#} \textcolor{nocommand}{no service dhcp}
\end{Verbatim}
\end{enumerate}
\subsection{DHCPv6}



\chapter{Redundancy}
\section{Etherchannel}
\begin{itemize}
    \item \textbf{Following config must be same on both switches for all etherchannel interfaces:}
    \begin{itemize}
        \item Switchport mode (trunk).
        \item Speed.
        \item Duplex.
        \item Allowed VLANs.
    \end{itemize}
\end{itemize}
\subsection{Topology}
\begin{verbatim}
+----+Fa0/1------Fa0/1+----+
| S1 |                | S2 |
+----+Fa0/2------Fa0/2+----+
\end{verbatim}
\subsection{LACP}
\begin{enumerate}
    \item \textbf{Configure LACP etherchannel:}
\begin{Verbatim}[commandchars=\\\{\}]
\textcolor{device}{Device(config)#} \textcolor{command}{interface} range <FA0/1 - 2>
\textcolor{device}{Device(config-if-range)#} \textcolor{command}{channel-group} <1> mode active
\textcolor{device}{Device(config)#} \textcolor{command}{interface} port-channel <1>
\textcolor{device}{Device(config-if)#} \textcolor{command}{switchport trunk encapsulation dot1q}
\textcolor{device}{Device(config-if)#} \textcolor{command}{switchport mode} <trunk>
\end{Verbatim}
\end{enumerate}



\chapter{Security}
\section{LAN Security}
\section{ACL}
\begin{itemize}
    \item \textbf{Standard ACLs -} source Layer 3 filtering. Range 1 to 99.
    \item \textbf{Extended ACLs -} source/destination Layer 3/4 filtering. Range 100 to 199.
\end{itemize}
\subsection{Numbered Standard ACL}
\begin{enumerate}
    \item \textbf{Create numbered standard ACL:}
\begin{Verbatim}[commandchars=\\\{\}]
\textcolor{device}{Router(config)#} \textcolor{command}{access-list} <1> [remark <DESCRIPTION>] <permits|deny>
\end{Verbatim}
\end{enumerate}


\chapter{Misc}
\section{Domain Lookup}
Domain lookup function interprets invalid commands as DNS names.
\begin{itemize}
    \item \textbf{Disable domain lookup:}
\begin{Verbatim}[commandchars=\\\{\}]
\textcolor{device}{Device(config)#} \textcolor{command}{no ip domain lookup}
\end{Verbatim}
    \item \textbf{Enable domain lookup:}
\begin{Verbatim}[commandchars=\\\{\}]
\textcolor{device}{Device(config)#} \textcolor{nocommand}{ip domain lookup}
\end{Verbatim}
\end{itemize}

\chapter{Linux Client Notes}
\section{Serial Connection}
\begin{enumerate}
    \item \textbf{Dependencies:}
\begin{Verbatim}[commandchars=\\\{\}]
\textcolor{root}{root#} \textcolor{command}{apt} install putty
\end{Verbatim}
    \item \textbf{Run as root:}
\begin{Verbatim}[commandchars=\\\{\}]
\textcolor{root}{root#} \textcolor{command}{putty}
\end{Verbatim}
    \item \textbf{Select Serial Interface}
\begin{itemize}
    \item RS232 = \textbf{/dev/ttyS0}
    \item USB console = \textbf{/dev/ttyUSB0}
\end{itemize}
\end{enumerate}
\section{SSH client}
\begin{itemize}
    \item \textbf{Choose older cipher and KeyExchange Algorithm:}
\end{itemize}
\begin{Verbatim}[commandchars=\\\{\}]
\textcolor{user}{ssh} <USER>@<HOST> -c <3des-cbc> -oKexAlgorithms=<+diffie-hellman-group1-sha1>
\end{Verbatim}





\chapter{ToDo}
\begin{itemize}
    \item DHCP6
    \item LAN Security
    \item OSPF
    \item ACL
    \item NAT
    \item NTP, Syslog, SNMP, CDP, LLDP
\end{itemize}



\end{document}
